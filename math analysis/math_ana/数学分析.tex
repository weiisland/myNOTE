\documentclass{ctexart}
\usepackage{graphicx}
\usepackage{amsmath}
\usepackage{mathtools}
\usepackage{bm}
\usepackage{amssymb}
\usepackage{xcolor}
\begin{document}
	\section{极限}
	\subsection{有界序列与无穷小序列}
	
	从自然数集N到实数集R的一个映射:
	
	\[x: N \to R\]
	
	相当于用自然数编号的一串实数
	
	\[x_1 = x(1), x_2 = x(2),...,x_n=x(n),...\]
	
	这样的一个映射,或者说这样的用自然数编号的一串实数\(\{x_n\}\),称为是一个实数序列
	
	1.有界序列
	
	\textbf{定义1:}设\(\{x_n\}\)是一个实数序列
	
	(1)如果存在\(M \in R\)使得:
	
	\[x_n \leq M, \forall n \in N\]
	
	就说序列\(\{x_n\}\)有上界,实数M是它的一个上界;
	
	(2)如果存在\(m \in R\)使得:
	
	\[x_n \geq m, \forall n \in N\]
	
	就说序列\(\{x_n\}\)有下界,实数m是它的一个下界;
	
	(3)如果序列\(\{x_n\}\)有上界也有下界,就说序列有界。
	
	序列设\(\{x_n\}\)有界的充分必要条件是:存在\(K \in R\)使得
	
	\[|x_n| \leq K, \forall n \in N\]
	
	例1:序列\(x_n=\frac{n+1}{n},(n=1,2,...)\)是有界的,
	
	\[\because |x_n|=|\frac{n+1}{n}| \leq |\frac{n+n}{n}| = 2, \forall n \in N\]
	
	例2:序列\(x_n=(1+\frac{1}{n})^n,(n=1,2,...)\)是有界的,因为
	
	\begin{align*}
	0 < x_n &= 1+n\frac{1}{n}+\frac{n(n-1)}{2!}\frac{1}{n^2}+\frac{n(n-1)(n-2)}{3!}\frac{1}{n^3} \\
	& +····+\frac{n(n-1)···(n-k+1)}{k!}\frac{1}{n^k}+···+\frac{1}{n^n} \\
	&= 1+1+\frac{1}{2!}(1-\frac{1}{n})+\frac{1}{3!}(1-\frac{1}{n})(1-\frac{2}{n}) \\
	&+···+\frac{1}{k!}(1-\frac{1}{n})···(1-\frac{k-1}{n})+···+\frac{1}{n!}(1-\frac{1}{n})···(1-\frac{n-1}{n}) \\
	&\leq 1+1+\frac{1}{2!}+\frac{1}{3!}+···+\frac{1}{k!}+···+\frac{1}{n!} \\
	& \leq 1+1+\frac{1}{2}+\frac{1}{2^2}+···+\frac{1}{2^{k-1}}+····+\frac{1}{2^{n-1}} \\
	&= 1+\frac{1-(\frac{1}{2})^n}{1-\frac{1}{2}} < 1+\frac{1}{1-\frac{1}{2}} = 3 \\
	\end{align*}
	
	\textbf{注意:}\(\frac{1}{n^n} = \frac{1}{n!}(1-\frac{1}{n})···(1-\frac{n-1}{n})\)
	
	例3:考察序列
	
	\[x_n=1+1/2+···+1/n, n=1,2,...\]
	
	证明这个序列无界。事实上,对任意自然数,只要取\(n=2^{2N}\),就有
	
	\begin{align*}
	x_n &= 1+\frac{1}{2}+(\frac{1}{3}+\frac{1}{4})+(\frac{1}{5}+\frac{1}{6}+\frac{1}{7}+\frac{1}{8})+···+(\frac{1}{2^{k-1}+1}+···+\frac{1}{2^k}) \\
	&+···+(\frac{1}{2^{2N-1}+1}+···+\frac{1}{2^{2N}}) \\
	& > 1+\frac{1}{2}+4\frac{1}{2^3}+···+2^{k-1}\frac{1}{2^k}+···+2^{2N-1}\frac{1}{2^{2N}} \\
	& > \frac{1}{2}+\frac{1}{2}+···+\frac{1}{2} = 2N\frac{1}{2}=N\\
	\end{align*}
	
	
	
\end{document}