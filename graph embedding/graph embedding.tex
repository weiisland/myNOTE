\documentclass{ctexart}
\usepackage{graphicx}
\usepackage{xcolor}
\usepackage{amsmath}
\usepackage[colorlinks, linkcolor=red]{hyperref}
\begin{document}
	
	\section{Graph analytic tasks:}
	
	\subsection{node classification:}
	
	aims at determining the {\color{red}label of nodes} based on other labeled nodes and the topology(拓扑) of the network.
	
	\textbf{methods: }random walks to propagate the labels; extracts feature from nodes and apply classifier on them.
	
	\subsection{link prediction:}
	
	refers to the task of predicting missing links or links that are likely to occur in the future.
	
	\textbf{methods: }similarity based methods, maximum liklihood models, probability models.
	
	\subsection{clustering}
	
	find subsets of similar nodes and group them together.
	
	\textbf{methods: }attribute based models and methods which direclty maximize the inter-cluster distances.
	
	\subsection{visualization}
	
	helps in providing insights into the structure of the network.
	
	\section{definitions}
	
	\textbf{definition 1: }(graph) A graph G(V, E)is a collection of \(V = \{v_1,,,,v_n\}\) vertices and \(E = \{e_{ij}\}^n_{i,j=1}\)edges. The adjacency matrix S of graph G contains non-negative weights associated with each edge:\(s_{ij} \geq 0\).
	
	The edge weight \(s_{ij}\) is generally treated as a measure of similarity between the nodes \(v_i\) and \(v_j\). {\color{red}The higher the edge weight, the more similar the two nodes are expected to be.}
	
	
	\textbf{definition 1:}(Graph)
	
	
\end{document}