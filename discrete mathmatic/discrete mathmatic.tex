\documentclass{ctexart}
\usepackage{graphicx}
\usepackage{amsmath}
\usepackage{bm}
\usepackage{xcolor}
\usepackage{mathtools}
\begin{document}
	\section{关系}
	
	\subsection{关系及其性质}
	
	\textbf{定义1:}设A和B是集合,一个从A到B的二元关系是A X B的子集。
	
	换句话说,一个从A到B的二元关系是有序对的集合R,其中每一个有序对的第一个元素取自A并且第二个元素取自B。我们使用记号aRb表示\((a,b)\in R\),。当(a,b)属于R时,称a与b有关系R。
	
	\textbf{函数作为关系:}一个从集合A到集合B的函数f对于A中的每个元素都指定B中一个唯一的元素。
	
	\textbf{定义2:}集合A的关系是从A到A的关系
	
	换句话说,集合A的关系是A X A的子集。
	
	\textbf{关系的性质:}
	
	\textbf{定义3:}如果对每个元素\(a \in A\)有\((a, a) \in R\),那么集合A上的关系R叫做自反的
	
	如果A的每个元素都关系到它自己,A上的关系就是自反的(也就是说对每一个元素,自己与自己的关系也可以成立)
	
	\textbf{定义4:}对于\(a, b \in A\),如果只要\((a, b) \in R\)就有\((b,a) \in R\),则集合A上的关系R叫做对称。如果对于\(a, b \in A\),仅当a=b时\((a, b) \in R\)就有\((b,a) \in R\),则集合A上的关系R叫做反对称。
	
	\textbf{定义5:}如果对于\(a, b, c \in R, (a, b) \in R, (b,c) \in R\)则\((a, c)\in R\),那么集合A上的关系R叫做传递的
	
	\textbf{定义6:}设R是从集合A到集合B的关系,S是从集合B到集合C的关系。R和S的合成是由有序对(a,c)构成的关系。其中\(a \in A, c \in C\),并且对于它们存在一个元素\(b \in B\)使得\((a, b)\in R, (b,c)\in S\).用\(S \circ R\)表示R与S的合成。其中在R中有序对的第二元素与S中有序对的第一元素相同。
	
	\textbf{定义7:}设R是集合A上的关系。幂\(R^n, n=2,3,4...\)递归的定义为
	
	\[R^1 = R, R^{n+1} = R^n \circ R\]
	
	\textbf{定理1:}集合A上的关系R是传递的,当且仅当对n=1,2,3,有\(R^n \subseteq R\)
	
	\subsection{n元关系及其应用}
	
	\textbf{定义1:}设\(A_1,A_2,...,A_n\)是集合。在这些集合上的n元关系是\(A_1 X A_2 X...X A_n\)的子集。这些集合\(A_1,A_2,...,A_n\)叫做关系的域,n叫做它的阶
	
	\subsection{关系的表示}
	
	\textbf{用矩阵表示关系:}
	
	关系R可以用矩阵\(M_R = [m_{ij}]\)来表示,其中
	
	\[m_{ij} = 
	\begin{dcases}
	1,  & if (a_i, bj)\in R \\
	0, & if (a_i, b_j) \notin R
	\end{dcases}\]
	
	\(M_R\)的主对角线的所有元素都等于1,那么R是自反的
	
	\(M_R\)是对称矩阵,则R是对称的
	
	当矩阵\(i \neq j, m_{ij}=0 or m_{ji} = 0\),关系R是反对称的
	
	\textbf{用图表示关系}
	
	\textbf{定义1:}一个有向图由顶点集V和边集E组成,其中边集是V中元素的有序对的集合。顶点a叫做边(a,b)的始点,顶点b叫做这条边的终点。
	
	形如(a, a)的形式叫做环。
	
	一个关系是自反的,当且仅当又想吐的每个顶点都有环,从而每个形如(x , x)的有序对都出现在关系中
	
	一个关系是对称的,当且仅当对有向图不同顶点之间的每一条边都存在一条方向相反的边,从而只要(x,y)在关系中就有(y,x)在关系中
	
	一个关系是反对称的,当且仅当在不痛的两个顶点之间不存在两条方向相反的边。
	
	一个关系是传递的,当且仅当只要存在一条从顶点x到顶点y的边和一条顶点y到顶点z的边,就有一条从顶点x到顶点z的边。
	
	\subsection{关系的闭包}
	
	设R是集合A上的关系,R可能具有或者不具有某些性质P。如果存在包含R的具有性质P的关系S,并且S是包含R且具有性质P的每一个关系的子集,那么S叫做R的关于P的闭包。
	
	自反闭包:给定集合A上的关系R,对于\(a \in R\),可以通过把形如(a, a)的所有的对,除了已在R中的之外,都加入到R中,就构成了R的自反闭包。加入这些对产生了一个新的自反的,包含R的关系,并且它被包含在任何包含R的自反关系中。自反闭包等于\(R \cup \Delta, \Delta = \{(a, a)|a\in A\}\)
	
	对称闭包:关系R的对称闭包可以通过取关系与它的逆的并来构造,即\(R \cup R^{-1}, R^{-1}=\{(b,a)|(a,b)\in R\}\)
	
	
	
	
	
	
	
	
	
	
	
	
	
	
	
	
	
	
	
	
	
	
	
	
\end{document}